
\chapter*{} %% Abstract

El cálculo $\lambda_\rho$ introducido por Díaz-Caro en 2017 es un lenguaje basado en el lambda cálculo con extensiones para la computación cuántica donde se utiliza un modelo de control clásico y datos cuánticos. Los estados cuánticos se describen mediante matrices de densidad, que permite operar con estados cuánticos mixtos.

En esta tesina se provee un algoritmo para tipar $\lambda_\rho$. Utilizando este algoritmo probamos que el tipado es NP-completo bajo condiciones de minimización de los tipos. Seguidamente analizamos la relación entre $\lambda_\rho$ y las aplicaciones actuales de la computación cuántica al definir una traducción de una versión modificada de $\lambda_\rho$ a Python y haciendo uso de la biblioteca de Qiskit. Además de probar su correctitud, implementamos estos algoritmos en el lenguaje funcional Haskell.

\textbf{Palabras claves:} Lambda cálculo, computación cuántica, matrices de densidad, control clásico, Qiskit, Python, traducción, inferencia de tipos, Haskell, compilador, IBM, cubrimiento de vértices mínimo, NP-completo, NP-hard, NP-difícil, Simplex, Ramificación y Poda.


\chapter*{} %% Abstract
The $\lambda_\rho$ calculus introduced by Díaz-Caro in 2017 is a language based on the lambda calculus with extensions for quantum computing where a classical control with quantum data model is used. Quantum states are described by density matrices, which allow mixed quantum states to be used.

In this thesis we first provide a type inference algorithm to type $\lambda_\rho$. Using this algorithm, we then prove that typing this language is NP-complete under minimization of the size of the types. Then we analyse the relationship between $\lambda_\rho$ and current applications of quantum computing by defining a translation between a modified version of $\lambda_\rho$ and Python, while making use of the Qiskit library. In addition to proving its correctness, we implement these algorithms using the Haskell functional language.


\textbf{Keywords:} Lambda calculus, quantum computing, density matrices, classic control, Qiskit, Python, translation, type inference, Haskell, compiler, IBM, minimum vertex cover, NP-complete, NP-hard, NP-hardness, Simplex, Branch and Bound.