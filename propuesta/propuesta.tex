\documentclass[a4paper]{article}
\usepackage[utf8]{inputenc}
\usepackage[T1]{fontenc}
\usepackage[spanish]{babel}
\usepackage{url}

\usepackage{color,soulutf8}
\newcommand\todo{\hl{TO-DO:} }

\title{
  {\normalsize\sc Propuesta de Tesina\\ de Licenciatura en Ciencias de la
    Computación}\\
  Compilación de $\lambda$-cálculo con matrices densidad\\
  en la máquina cuántica IBM-Q
}
\author{
  Martín Villagra\\
  V-2719/7\\
  {\tt mvillagra0@gmail.com}
}

\date{\today}

\newcommand\LambdaS{\textsf{Lambda-}$\mathcal S$}
\newcommand\Lineal{\textsf{Lineal}}

\begin{document}
\maketitle
\noindent  Director: Alejandro Díaz-Caro\\
Investigador Adjunto en el Instituto de Ciencias de la Computación
(UBA/CONICET)\\
Profesor Adjunto en la Universidad Nacional de Quilmes\\
{\tt adiazcaro@icc.fcen.uba.ar}\\[1ex]
Codirector: Pablo E. Martínez López\\
Profesor Titular en la Universidad Nacional de Quilmes\\
{\tt fidel@unq.edu.ar}

% https://dcc.fceia.unr.edu.ar/es/lcc/tesinas-grado
\section*{Introducción y estado del arte}
La computación cuántica se puede entender como un modelo de cómputo novedoso,
basado en las propiedades de la física, o como un formalismo matemático para
describir la física cuántica. Desde el punto de vista de la tecnología, las
computadoras cuánticas son dispositivos programables, y por tanto, existen
diversos lenguajes de programación para las mismas. El más común, se basa en
circuitos cuánticos (el análogo a los circuitos electrónicos). Desde el punto de
vista más teórico, el estudio de fundamentos de lenguajes de programación para
las computadoras cuánticas, no sólo tiene por objetivo el desarrollo de dichos
lenguajes, sino también el estudio de propiedades de la física cuántica,
puramente desde el formalismo matemático, en particular dada la conexión entre
los fundamentos de lenguajes (y la teoría de tipos), con la lógica.

En esta línea, existen diversas extensiones al cálculo lambda, y diversos
sistemas de tipos, para lidiar con diferentes características de la computación
cuántica. En particular, existen dos paradigmas muy diferenciados.

Por un lado, está el paradigma de ``control clásico''~\cite{SelingerMSCS04}, el
cual introduce en el lenguaje la posibilidad de describir circuitos cuánticos.
Dichas descripciones son clásicas: las compuertas cuánticas se describen como
cajas negras, y el flujo de control que dice qué compuerta aplicar a qué qubit,
es clásico, en el sentido de que no existe superposición de flujos de control, o
de programas. Dicho paradigma está justificado por el hecho de que las
computadoras cuánticas se visualizan como dispositivos anexos a las computadoras
clásicas, y son estas últimas las que instruyen a las primeras sobre las
operaciones a realizar. Así, para programar, por ejemplo, la computadora
cuántica IBM-Q\footnote{\url{https://www.ibm.com/quantum-computing/}}, con su
SDK Qiskit\footnote{\url{https://qiskit.org}} (un conjunto de librerías de
Python), se desarrolla un programa clásico en Python, con algunas instrucciones
extra para instruir a la computadora cuántica qué circuito cuántico debe
realizar. El código Qiskit es compilado internamente a QAsm~\cite{qiskit_specifications._2018} que puede correr tanto en la computadora cuántica, como
en un simulador en una computadora clásica.

Un paradigma alternativo es el de ``control
cuántico''~\cite{AltenkirchGrattageLICS05,ArrighiDowekLMCS17,DiazcaroGuillermoMiquelValironLICS19}.
La idea es que si queremos superar la etapa de circuitos (de la misma manera que
lenguajes de alto nivel superaron la etapa de los circuitos electrónicos),
debemos poder describir operaciones que, al ser compiladas, se transformen en
circuitos cuánticos. Lo que se pretende es abrir la ``caja negra'' que
representa a las operaciones cuánticas, y que la misma pueda ser descripta en el
lenguaje. Para ello el lenguaje tiene que tener características cuánticas, como
permitir la superposición de programas, o el enredo de los mismos. Sin embargo,
el lenguaje deberá proveer una herramienta que asegure que lo que se escriba en
el mismo, tenga una traducción (una compilación) directa a la computadora
cuántica.

Un paradigma intermedio, llamado ``control
probabilístico''~\cite{DiazcaroAPLAS17}, consiste en utilizar el formalismo de
las matrices densidad, que permite representar mezclas probabilísticas de
estados cuánticos puros, generalizándolo a mezclas probabilísticas de programas.
Las matrices densidad no son utilizadas usualmente para describir algoritmos,
donde el output esperado es el resultado clásico de medir un estado cuántico.
Sin embargo, el formalismo es utilizado ampliamente en teorías como la
decoherencia cuántica. Desde el punto de vista de la computación cuántica como
una descripción de la física, estudiar el paradigma del control probabilístico,
y posibles simulaciones del mismo, es una herramienta más en el estudio de la
mecánica cuántica.

En~\cite{DiazcaroAPLAS17} se presentan dos extensiones al lambda cálculo para
computación cuántica: $\lambda_\rho$ y $\lambda_\rho^\circ$. Ambas utilizan
matrices densidad, sólo que en la primera, $\lambda_\rho$, si la matriz densidad representa a un
estado puro al inicio del cálculo, terminará en un estado puro al finalizar, y,
por lo tanto, $\lambda_\rho$ está en el paradigma del control clásico. En
cambio, la segunda extensión, $\lambda_\rho^\circ$, introduce la noción de
matriz densidad generalizada, dando posibilidad de llevar la evolución de
estados mixtos. Por lo tanto, $\lambda_\rho^\circ$ está en el paradigma de
control probabilístico.


\section*{Objetivos y actividades}
El objetivo de esta tesina es definir e implementar un compilador de
$\lambda_\rho$ en código Qiskit, que corra tanto en el simulador como en la
computadora cuántica IBM-Q, y definir e implementar un compilador de
$\lambda_\rho^\circ$ junto con un simulador del mismo que utilice matrices de densidad generalizadas.
\\
\\
\\
\newline
\newline
\newline
\newline
\newline
\newline
Para esto, se definen las siguientes actividades.
\begin{enumerate}
\item Definir una traducción desde $\lambda_\rho$ a
  Python que utilice la librería Qiskit. Parte del trabajo es explorar las técnicas disponibles y escoger las convenientes.
\item Programar un compilador en Haskell que siga la traducción especificada en el paso anterior. El objetivo es que el código generado pueda ser simulado/ejecutado utilizando la librería Qiskit.
\item Desarrollar una herramienta que permita simular los programas en $\lambda_\rho^\circ$ siguiendo la semántica especificada en ~\cite{DiazcaroAPLAS17}, mediante el uso de matrices de densidad. Esto involucra en primer medida la implementación de un compilador que procese el código de $\lambda_\rho^\circ$.
\end{enumerate}

\bibliographystyle{alpha}
\bibliography{biblio} 
\end{document}
